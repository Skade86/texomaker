%@ Titre: Relèvement d'un facteur de puissance
%@ Domaine: Electronique
%@ Chapitre: Révisions
%@ figure: fig-exemple-1

Considérons un moteur d’impédance $\underline{Z} = R + jX$ à caractère inductif ($X > 0$). On souhaite relever le facteur de puissance de ce réseau, c’est-à-dire donner à $\cos\varphi$ une valeur égale à l’unité sans dépense d’énergie.

\begin{figure}[h]
 \centering
\includegraphics[width=7cm]{fig-exemple-1}
 \end{figure}

\begin{enumerate}
\item Calculer, en fonction de $R$, $X$ et $\omega$, la capacité $C_1$ à placer en parallèle sur le réseau pour que le facteur de puissance devienne égal à 1.
\item Quelle capacité $C_2$ aurait-il fallu placer en série sur le réseau pour obtenir le même résultat ?
\item Des deux solutions, quelle est celle à retenir si le moteur fonctionne sur le secteur ?
\item On considère un moteur ($\cos\varphi = 0,7$) de puissance $\mathcal{P} = 10\,$kW alimenté sous une tension de fréquence $f = 50\,$Hz et d’amplitude $U_m = 220\sqrt{2}\,$V. On souhaite relever à 1 son facteur de puissance à l’aide d’une batterie de condensateurs de capacité $C$, placée en parallèle avec le moteur.
\begin{enumerate}
\item  Calculer les intensités efficaces $I_{eff}$ et $I'_{eff}$ traversant le circuit d’alimentation avant et après le relèvement du facteur de puissance.
\item  Quelle est l’intérêt du relèvement de puissance sur les pertes en ligne par effet Joule (énergie dissipée sous forme de chaleur dans la ligne pour amener la puissance à l’installation) ?
\item  Calculer la valeur de la résistance $R$ du moteur (qu’on modélisera par une impédance complexe $\underline{Z} = R + jX$).
\item  Déterminer l’expression de $\tan\varphi$ en fonction de $R$ et $X$, et en déduire la valeur de la capacité $C_1$ à placer en parallèle pour relever le facteur de puissance de l’installation.
\end{enumerate}
\end{enumerate}

 
